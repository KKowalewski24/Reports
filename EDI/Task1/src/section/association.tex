\documentclass[../EDI_Task1_Karwowski_Kowalewski.tex]{subfiles}

\begin{document} {

    Do odnalezienia reguł asocjacyjnych został użyty algorytm Apriori, który należy
    do grupy algorytmów interacyjnych. Polega on na iteracyjnym wyszukiwaniu coraz
    większych zbiorów elementów często występujących. Sam proces jest rozpoczynany
    od znalezienia zbiorów jednoelementowych o odpowiednim wsparciu. W kolejnej
    iteracji na ich podstawie są tworzone zbiory kandydujące dwuelementowe i na
    podstawie ich wsparcia, gdy jest ono wystarczająco są tworzone zbiory
    trójelementowe i proces iteracyjnie podążą dalej aż do stanu gdy nie da się
    wygenerować kolejnych zbiorów kandydujących.

    Algorytm ten został uruchomiony dwukrotnie. Po raz pierwszy dla całego pliku sesji,
    gdzie wartości liczbowe zostały zdyskretyzowane. Po raz drugi tylko dla tych
    atrybutów, które na początku miały wartości liczbowe ciągłe, a po dyskretyzacji
    wartości dyskretne.

    \begin{table}[!htbp]
        \footnotesize
        \centering
        \begin{tabular}{|l|c|}
            \hline
            Parametr & Wartość \\ \hline
            Minimum support & 0.95 (5120 instances) \\
            Minimum metric <confidence> & 0.9 \\
            Number of cycles performed & 1 \\
            Size of set of large itemsets L(1) & 23 \\
            Size of set of large itemsets L(2) & 103 \\
            Size of set of large itemsets L(3) & 154 \\
            Size of set of large itemsets L(4) & 82 \\
            Size of set of large itemsets L(5) & 15 \\ \hline
        \end{tabular}
        \caption
        {Parametry przebiegu algorytmu Apriori dla całego pliku sesji}
        \label{apriori_params}
    \end{table}

    \begin{table}[!htbp]
        \scriptsize
        \centering
        \begin{tabular}{|l|l|}
            \hline
            Jeśli & To \\ \hline
            /facilities/tour.html=False 5248 & /images/=False 5235\\
            /software/winvn/winvn.html=False /facilities/tour.html=False 5233 & /images/=False 5220\\
            /facts/faq04.html=False /facilities/tour.html=False 5229 & /images/=False 5216\\
            /facts/faq04.html=False /software/winvn/winvn.html=False /facilities/tour.html=False 5214 & /images/=False 5201\\
            /facilities/tour.html=False /elv/elvpage.htm=False 5175 & /images/=False 5162\\
            /software/winvn/winvn.html=False /facilities/tour.html=False /elv/elvpage.htm=False 5160 & /images/=False 5147\\
            /facilities/tour.html=False /shuttle/missions/sts-69/images/images.html=False 5158 & /images/=False 5145\\
            /facts/faq04.html=False /facilities/tour.html=False /elv/elvpage.htm=False 5157 & /images/=False 5144\\
            /facilities/tour.html=False /shuttle/countdown/lps/fr.html=False 5149 & /images/=False 5136\\
            /procurement/procurement.html=False /facilities/tour.html=False 5145 & /images/=False 5132\\ \hline
        \end{tabular}
        \caption
        {Uzyskane reguły asocjacyjne dla całego pliku sesji}
        \label{asociate_rules}
    \end{table}
    \FloatBarrier

     \begin{table}[!htbp]
        \footnotesize
        \centering
        \begin{tabular}{|l|c|}
            \hline
            Parametr & Wartość \\ \hline
           Minimum support & 0.1 (539 instances) \\
            Minimum metric <confidence> & 0.9 \\
            Number of cycles performed & 18 \\
            Size of set of large itemsets L(1) & 5 \\
            Size of set of large itemsets L(2) & 6 \\
            Size of set of large itemsets L(3) & 2 \\ \hline
        \end{tabular}
        \caption
        {Parametry przebiegu algorytmu Apriori dla atrybutów liczbowych z pliku sesji}
        \label{apriori_params_numeric}
    \end{table}

    \begin{table}[!htbp]
        \footnotesize
        \begin{enumerate}
            \item requests\_count='(-inf-7.9]' average\_request\_duration='(-inf-59.9]' 1946 ==> duration='(-inf-709.8]' 1946 <conf:(1)>
            \item requests\_count='(-inf-7.9]' average\_request\_duration='(59.9-119.8]' 1046 ==> duration='(-inf-709.8]' 1042 <conf:(1)>
            \item average\_request\_duration='(-inf-59.9]' 2154 ==> duration='(-inf-709.8]' 2126 <conf:(0.99)>
            \item duration='(-inf-709.8]' 4748 ==> requests\_count='(-inf-7.9]' 4474 <conf:(0.94)>
            \item requests\_count='(-inf-7.9]' 4765 ==> duration='(-inf-709.8]' 4474 <conf:(0.94)>
            \item duration='(-inf-709.8]' average\_request\_duration='(59.9-119.8]' 1136 ==> requests\_count='(-inf-7.9]' 1042 <conf:(0.92)>
            \item duration='(-inf-709.8]' average\_request\_duration='(-inf-59.9]' 2126 ==> requests\_count='(-inf-7.9]' 1946 <conf:(0.92)>
            \item average\_request\_duration='(-inf-59.9]' 2154 ==> requests\_count='(-inf-7.9]' 1946 <conf:(0.9)>
            \item average\_request\_duration='(-inf-59.9]' 2154 ==> duration='(-inf-709.8]' requests\_count='(-inf-7.9]' 1946 <conf:(0.9)>
        \end{enumerate}
    \caption{Uzyskane reguły asocjacyjne dla atrybutów liczbowych z pliku sesji}
    \end{table}
    \FloatBarrier

    Jak widać reguły asocjacyjne dla całego pliku sesji nie zawierają przydatnych
    informacji. Wynika to z faktu, że w bardzo wielu sesjach można łatwo znaleźć
    zależności, które są oparte na samych atrybutach flagowych. Jednakże tego typu
    zależności nie niosą przydatnej informacji.

    Inna sytuacja ma miejsce w przypadku szukania reguł asocjacyjnych dla samych danych
    liczbowych (zdyskretyzowanych). Udało się tutaj znaleźć pewne interesujące zależności.
    Przykładowo dla sesji składających z krótkim średnim czasem na stronie, czas całej
    sesji też jest krótki - jest to zależność jak najbardziej logiczna. Inny przykład to,
    że dla krótkich sesji mają one zazwyczaj mało zapytań.
}
\end{document}
