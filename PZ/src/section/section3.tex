\documentclass[../main.tex]{subfiles}

\begin{document} {
    Z~roku na rok miasta coraz bardziej się rozrastają a~co za tym idzie powstaje duża
    liczba bloków mieszkalnych oraz domów wolnostojących aby zapewnić ludzią miejsce do
    życia. Wedle badań Głównego Urzędu Statystycznego w~2019 roku powstało 197,7 tys
    nowych mieszkań\cite{gospodarka_mieszkaniowa}. W~celu zapewnienia bezpieczeństwa
    montowane są zamki w~drzwiach. Aby je otworzyć wymagane są do nich unikalne
    klucze lub skorzystanie z~na przykład domofonu. W~nawiązaniu do opisu rozwiązań
    konkurencyjnych przedstawionych w~sekcji
    \ref{chapter1:opis_produktu:rozw_konkurencyjne} warto dodać, że aktualnie dostępne
    rozwiązania posiadają stosunkowo dużo wad szczególnie dla osób zapominalskich gdyż
    w~sytuacji gdy nie ma żadnego domownika a~nie mamy kluczy to dostanie się do
    mieszkania jest niemożliwe.

    Na samym rynku tego typu urządzeń jest bardzo szeroki wybór, poprzez
    podstawowe domofony niskiej jakości za bardzo małą ilość pieniędzy aż do bardzo
    drogich i~rozbudowanych systemów za pokaźne kwoty pieniędzy. Przykładem
    najprostszego i~najtańszego domofonu jest domofon
    \textit{Unifon analogowy Cyfral ADA-03C4 Slim czarno-biały}, którego cena wynosi
    aktualnie około \textit{56 złotych}\cite{tani_domofon}. Przechodząc do najbardziej
    zaawansowanego a~co za tym idzie najdroższego domofonu z~dostępnych w~sieci sklepów
    budowlanych Castorama czyli do \textit{Wideodomofon Somfy V300 biały}, którego cena
    wynosi aż \textit{998 złotych}\cite{drogi_domofon}. Nie trudno zauważyć potężną
    dysproporcje między tymi urządzeniami

    Warto zauważyć dosyć poważną lukę w~rynku gdyż nie ma podobnego rozwiązania,
    które byłoby aktualnie dostępne. Sama koncepcja jest bardzo nowatorska i~może
    przyciągnąć wiele potencjalnych klientów zainteresowanych nowinkami
    technologicznymi jak i~przeciętnych zjadaczy chleba chcąc ułatwić sobie codzienne
    życie.
}
\end{document}
