\documentclass[../EDI_Task3_Karwowski_Kowalewski.tex]{subfiles}

\begin{document} {

    Do przeprowadzenie poniższych eksperymentów został wygenerowany losowy użytkownik i
    był użyty w niezmienionej formie we wszystkich eksperymentach. Został on
    przedstawiony poniżej:

% @formatter:off
    \begin{lstlisting}
False, False, False, False, False, False, True, False, True, True,
False, False, True, True, True, True, True, False, False, False,
True, True, False, False, True, False, False, True, True, False,
False, False, False, True, True, False
    \end{lstlisting}
% @formatter:on

    \subsection{4 klastry}
    \label{results_k_4} {

    % @formatter:off
        \begin{table}[!htbp]
            \scriptsize
            \centering
            \begin{tabular}{|c|c|c|c|c|}
                \hline
                Klaster 0 & Klaster 1 & Klaster 2 & Klaster 3 \\ \hline
                  True   &   False   &   False   &   False \\
                 False   &   False   &    True   &   False \\
                 False   &   False   &   False   &   False \\
                 False   &   False   &   False   &   False \\
                 False   &   False   &   False   &   False \\
                 False   &   False   &   False   &   False \\
                 False   &   False   &   False   &   False \\
                 False   &   False   &   False   &   False \\
                  True   &   False   &   False   &   False \\
                  True   &   False   &   False   &   False \\
                 False   &   False   &   False   &    True \\
                 False   &   False   &   False   &   False \\
                 False   &   False   &   False   &   False \\
                 False   &   False   &   False   &   False \\
                 False   &   False   &   False   &   False \\
                 False   &   False   &   False   &   False \\
                 False   &   False   &   False   &   False \\
                 False   &   False   &   False   &   False \\
                 False   &   False   &   False   &   False \\
                 False   &   False   &   False   &   False \\
                 False   &   False   &   False   &   False \\
                 False   &   False   &   False   &   False \\
                 False   &   False   &   False   &   False \\
                 False   &   False   &   False   &   False \\
                 False   &   False   &   False   &   False \\
                 False   &   False   &   False   &   False \\
                 False   &   False   &   False   &   False \\
                 False   &   False   &   False   &   False \\
                 False   &   False   &   False   &   False \\
                 False   &   False   &   False   &   False \\
                 False   &   False   &   False   &   False \\
                 False   &   False   &   False   &   False \\
                 False   &   False   &   False   &   False \\
                 False   &   False   &   False   &   False \\
                 False   &   False   &   False   &   False \\
                 False   &   False   &   False   &   False \\ \hline
            \end{tabular}
            \caption{Centroidy dla 4 klastrów z flagami stron użytkowników}
%            \label{}
        \end{table}
    % @formatter:on

        \begin{table}[!htbp]
            \small
            \centering
            \begin{tabular}{|c|c|c|}
                \hline
                Klaster & Liczba obserwacji & Procent obserwacji \\ \hline
                0 & 294 & 4\% \\
                1 & 6158 & 77\% \\
                2 & 1218 & 15\% \\
                3 & 336 & 4\% \\ \hline
            \end{tabular}
            \caption{Wyniki analizy skupień dla 4 klastrów}
%            \label{}
        \end{table}

        \begin{table}[!htbp]
            \small
            \centering
            \begin{tabular}{|c|c|}
                \hline
                Klaster & Współczynnik Jaccarda \\ \hline
                0 & \textbf{0.125} \\
                1 & 0.0 \\
                2 & 0.0 \\
                3 & 0.0 \\ \hline
            \end{tabular}
            \caption{Wartości współczynnika podobieństwa Jaccarda}
            \label{coeffcient_5}
        \end{table}
        \FloatBarrier

        Po dokonaniu analizy została uzyskana informacja, że najlepiej dopasowanym
        klasterem jest klaster numer 0 z wartością współczynnika podobieństwa Jaccarda
        na poziomie 0.125 co zostało przedstawione w tabeli \ref{coeffcient_5}. W tym
        przypadku system dokonał rekomendacji następującej strony:
        \begin{itemize}
            \item \textbf{/ksc.html}
        \end{itemize}

    }

    \subsection{7 klastrów}
    \label{results_k_7} {

    % @formatter:off
        \begin{table}[!htbp]
            \scriptsize
            \centering
            \begin{tabular}{|c|c|c|c|c|c|c|}
                \hline
                Klaster 0 & Klaster 1 & Klaster 2 & Klaster 3 & Klaster 4 & Klaster 5 & Klaster 6 \\ \hline
                False   &   False   &   False   &   False   &    True   &   False   &   False \\
                False   &   False   &    True   &   False   &   False   &   False   &   False \\
                False   &   False   &   False   &   False   &   False   &    True   &   False \\
                False   &    True   &   False   &   False   &   False   &   False   &   False \\
                False   &   False   &   False   &   False   &   False   &   False   &   False \\
                False   &   False   &   False   &   False   &   False   &   False   &   False \\
                False   &   False   &   False   &   False   &   False   &   False   &   False \\
                False   &   False   &   False   &   False   &   False   &   False   &   False \\
                 True   &   False   &   False   &   False   &   False   &   False   &   False \\
                 True   &   False   &   False   &   False   &   False   &   False   &   False \\
                 True   &   False   &   False   &    True   &   False   &   False   &   False \\
                False   &   False   &   False   &   False   &   False   &   False   &   False \\
                False   &   False   &   False   &   False   &   False   &   False   &   False \\
                False   &   False   &   False   &   False   &   False   &   False   &   False \\
                False   &   False   &   False   &   False   &   False   &   False   &   False \\
                False   &   False   &   False   &   False   &   False   &   False   &   False \\
                False   &   False   &   False   &   False   &   False   &   False   &   False \\
                False   &   False   &   False   &   False   &   False   &   False   &   False \\
                False   &   False   &   False   &   False   &   False   &   False   &   False \\
                False   &   False   &   False   &   False   &   False   &   False   &   False \\
                False   &   False   &   False   &   False   &   False   &   False   &   False \\
                False   &   False   &   False   &   False   &   False   &   False   &   False \\
                False   &   False   &   False   &   False   &   False   &   False   &   False \\
                False   &   False   &   False   &   False   &   False   &   False   &   False \\
                False   &   False   &   False   &   False   &   False   &   False   &   False \\
                False   &   False   &   False   &   False   &   False   &   False   &   False \\
                False   &   False   &   False   &   False   &   False   &   False   &   False \\
                False   &   False   &   False   &   False   &   False   &   False   &   False \\
                False   &   False   &   False   &   False   &   False   &   False   &   False \\
                False   &   False   &   False   &   False   &   False   &   False   &   False \\
                False   &   False   &   False   &   False   &   False   &   False   &   False \\
                False   &   False   &   False   &   False   &   False   &   False   &   False \\
                False   &   False   &   False   &   False   &   False   &   False   &   False \\
                False   &   False   &   False   &   False   &   False   &   False   &   False \\
                False   &   False   &   False   &   False   &   False   &   False   &   False \\
                False   &   False   &   False   &   False   &   False   &   False   &   False \\ \hline
            \end{tabular}
            \caption{Centroidy dla 7 klastrów z flagami stron użytkowników}
%            \label{}
        \end{table}
    % @formatter:on

        \begin{table}[!htbp]
            \small
            \centering
            \begin{tabular}{|c|c|c|}
                \hline
                Klaster & Liczba obserwacji & Procent obserwacji \\ \hline
                0 & 305 & 4\% \\
                1 & 1002 & 13\% \\
                2 & 1066 & 13\% \\
                3 & 246 & 3\% \\
                4 & 1208 & 15\% \\
                5 & 754 & 9\% \\
                6 & 3425 & 43\% \\ \hline
            \end{tabular}
            \caption{Wyniki analizy skupień dla 7 klastrów}
%            \label{}
        \end{table}

        \begin{table}[!htbp]
            \small
            \centering
            \begin{tabular}{|c|c|}
                \hline
                Klaster & Współczynnik Jaccarda \\ \hline
                0 & \textbf{0.125} \\
                1 & 0.0 \\
                2 & 0.0 \\
                3 & 0.0 \\
                4 & 0.0 \\
                5 & 0.0 \\
                6 & 0.0 \\ \hline
            \end{tabular}
            \caption{Wartości współczynnika podobieństwa Jaccarda}
            \label{coeffcient_7}
        \end{table}
        \FloatBarrier

        Po dokonaniu analizy została uzyskana informacja, że najlepiej dopasowanym
        klasterem jest klaster numer 0 z wartością współczynnika podobieństwa Jaccarda
        na poziomie 0.125 co zostało przedstawione w tabeli \ref{coeffcient_7}. W tym
        przypadku system dokonał rekomendacji następującej strony:
        \begin{itemize}
            \item \textbf{/history/apollo/apollo-13/apollo-13.html}
        \end{itemize}

    }

    \subsection{10 klastrów}
    \label{results_k_10} {

    % @formatter:off
        \begin{table}[!htbp]
            \scriptsize
            \centering
            \begin{tabular}{|c|c|c|c|c|c|c|c|c|c|c|}
                \hline
                Klaster 0 & Klaster 1 & Klaster 2 & Klaster 3 & Klaster 4 & Klaster 5 & Klaster 6 & Klaster 7 & Klaster 8 & Klaster 9 \\ \hline
                 True   &   False   &   False   &   False   &    True   &   False   &   False   &   False   &   False   &   False \\
                False   &   False   &    True   &   False   &   False   &   False   &   False   &   False   &   False   &   False \\
                False   &   False   &   False   &   False   &   False   &    True   &   False   &   False   &    True   &   False \\
                False   &    True   &   False   &   False   &   False   &   False   &   False   &   False   &   False   &   False \\
                False   &   False   &   False   &   False   &   False   &   False   &   False   &   False   &   False   &    True \\
                False   &   False   &   False   &   False   &   False   &   False   &   False   &   False   &   False   &   False \\
                False   &   False   &   False   &   False   &   False   &   False   &   False   &   False   &   False   &   False \\
                False   &   False   &   False   &   False   &   False   &   False   &   False   &   False   &   False   &   False \\
                 True   &   False   &   False   &   False   &   False   &   False   &   False   &   False   &   False   &   False \\
                 True   &   False   &   False   &   False   &   False   &   False   &   False   &    True   &   False   &   False \\
                 True   &   False   &   False   &    True   &   False   &   False   &   False   &   False   &   False   &   False \\
                False   &   False   &   False   &   False   &   False   &   False   &   False   &   False   &   False   &   False \\
                False   &   False   &   False   &   False   &   False   &   False   &   False   &   False   &    True   &   False \\
                False   &   False   &   False   &   False   &   False   &   False   &   False   &   False   &   False   &   False \\
                False   &   False   &   False   &   False   &   False   &   False   &   False   &   False   &   False   &   False \\
                False   &   False   &   False   &   False   &   False   &   False   &   False   &   False   &   False   &   False \\
                False   &   False   &   False   &   False   &   False   &   False   &   False   &   False   &   False   &   False \\
                False   &   False   &   False   &   False   &   False   &   False   &   False   &   False   &   False   &   False \\
                False   &   False   &   False   &   False   &   False   &   False   &   False   &   False   &   False   &   False \\
                False   &   False   &   False   &   False   &   False   &   False   &   False   &   False   &   False   &   False \\
                False   &   False   &   False   &   False   &   False   &   False   &   False   &   False   &   False   &   False \\
                False   &   False   &   False   &   False   &   False   &   False   &   False   &   False   &   False   &   False \\
                False   &   False   &   False   &   False   &   False   &   False   &   False   &   False   &   False   &   False \\
                False   &   False   &   False   &   False   &   False   &   False   &   False   &   False   &   False   &   False \\
                False   &   False   &   False   &   False   &   False   &   False   &   False   &   False   &   False   &   False \\
                False   &   False   &   False   &   False   &   False   &   False   &   False   &   False   &   False   &   False \\
                False   &   False   &   False   &   False   &   False   &   False   &   False   &   False   &   False   &   False \\
                False   &   False   &   False   &   False   &   False   &   False   &   False   &   False   &   False   &   False \\
                False   &   False   &   False   &   False   &   False   &   False   &   False   &   False   &   False   &    True \\
                False   &   False   &   False   &   False   &   False   &   False   &   False   &   False   &   False   &   False \\
                False   &   False   &   False   &   False   &   False   &   False   &   False   &   False   &   False   &   False \\
                False   &   False   &   False   &   False   &   False   &   False   &   False   &   False   &   False   &   False \\
                False   &   False   &   False   &   False   &   False   &   False   &   False   &   False   &   False   &   False \\
                False   &   False   &   False   &   False   &   False   &   False   &   False   &   False   &   False   &   False \\
                False   &   False   &   False   &   False   &   False   &   False   &   False   &   False   &   False   &   False \\
                False   &   False   &   False   &   False   &   False   &   False   &   False   &   False   &   False   &   False \\ \hline
            \end{tabular}
            \caption{Centroidy dla 10 klastrów z flagami stron użytkowników}
%            \label{}
        \end{table}
    % @formatter:on

        \begin{table}[!htbp]
            \small
            \centering
            \begin{tabular}{|c|c|c|}
                \hline
                Klaster & Liczba obserwacji & Procent obserwacji \\ \hline
                0 & 146 & 2\% \\
                1 & 991 & 12\% \\
                2 & 1060 & 13\% \\
                3 & 325 & 4\% \\
                4 & 1161 & 15\% \\
                5 & 657 & 8\% \\
                6 & 3232 & 40\% \\
                7 & 169 & 2\% \\
                8 & 147 & 2\% \\
                9 & 118 & 1\% \\ \hline
            \end{tabular}
            \caption{Wyniki analizy skupień dla 10 klastrów}
%            \label{}
        \end{table}

        \begin{table}[!htbp]
            \small
            \centering
            \begin{tabular}{|c|c|}
                \hline
                Klaster & Współczynnik Jaccarda \\ \hline
                0 & 0.1176 \\
                1 & 0.0 \\
                2 & 0.0 \\
                3 & 0.0 \\
                4 & 0.0 \\
                5 & 0.0 \\
                6 & 0.0 \\
                7 & 0.0667 \\
                8 & 0.0625 \\
                9 & 0.0625 \\ \hline
            \end{tabular}
            \caption{Wartości współczynnika podobieństwa Jaccarda}
            \label{coeffcient_10}
        \end{table}
        \FloatBarrier

        Po dokonaniu analizy została uzyskana informacja, że najlepiej dopasowanym
        klasterem jest klaster numer 0 z wartością współczynnika podobieństwa Jaccarda
        na poziomie 0.1176 co zostało przedstawione w tabeli \ref{coeffcient_10}. W tym
        przypadku system dokonał rekomendacji następujących stron:
        \begin{itemize}
            \item \textbf{/ksc.html}
            \item \textbf{/history/apollo/apollo-13/apollo-13.html}
        \end{itemize}
    }
}
\end{document}
