\documentclass{classrep}
\usepackage[utf8]{inputenc}
\frenchspacing

\usepackage{graphicx}
\usepackage[usenames,dvipsnames]{color}
\usepackage[hidelinks]{hyperref}
\usepackage{lmodern}
\usepackage{placeins}
\usepackage{url}
\usepackage{amsmath, amssymb, mathtools}
\usepackage{listings}
\usepackage{fancyhdr, lastpage}
\usepackage{subfiles}

\pagestyle{fancyplain}
\fancyhf{}
\renewcommand{\headrulewidth}{0pt}
\cfoot{\thepage\ / \pageref*{LastPage}}

\definecolor{codegreen}{rgb}{0,0.6,0}
\definecolor{codegray}{rgb}{0.5,0.5,0.5}
\definecolor{codepurple}{rgb}{0.58,0,0.82}
\definecolor{backcolour}{rgb}{0.95,0.95,0.92}

\lstdefinestyle{mystyle}{
    commentstyle=\color{codegreen},
    keywordstyle=\color{magenta},
    numberstyle=\tiny\color{codegray},
    stringstyle=\color{codepurple},
    basicstyle=\ttfamily\footnotesize,
    breakatwhitespace=false,
    breaklines=true,
    captionpos=b,
    keepspaces=true,
    numbers=left,
    numbersep=5pt,
    showspaces=false,
    showstringspaces=false,
    showtabs=false,
    tabsize=2
}

\lstset{style=mystyle}

%--------------------------------------------------------------------------------------%
\studycycle{Informatyka stosowana, studia dzienne, II st.}
\coursesemester{II}

\coursename{Eksploracja danych internetowych}
\courseyear{2021/2022}

\courseteacher{dr inż. Krzysztof Myszkorowski}
\coursegroup{poniedziałek, 12:15}

\author{%
    \studentinfo[239671@edu.p.lodz.pl]{Jan Karwowski}{239671}\\
    \studentinfo[239676@edu.p.lodz.pl]{Kamil Kowalewski}{239676}\\
}

\title{Zadanie 1.: Eksploracja użycia na podstawie pliku logów}

\begin{document}
    \maketitle
    \thispagestyle{fancyplain}

    \tableofcontents
    \newpage

    \section{Cel}
    \label{purpose} {
        Celem zadania było wstępne przetworzenie oraz analiza pliku z
        logami\cite{dataset}. Plik ten został dostarczonyw formacie CSV jednak
        była potrzeba wyodrebnienia oraz pogrupowania użytkowników oraz sesji aby
        można było wykorzystać te dane w programie Weka\cite{weka}. W samym programie Weka
        należało zbudować reguły asocjacyjnej dla wcześniej wspomnianych grup.
    }

    \section{Przygotowanie pliku z danymi} {

        \subsection{Opis zbioru danych} {
            Plik CSV o którym było juz mówione w sekcji \ref{purpose} zawiera 7 kolumn,
            ich opis znajduje sie poniżej:
            \begin{itemize}
                \item Bez nazwy - Identyfikator rekordu, swojego rodzaju ID
                \item host - nazwa hosta
                \item time - znaczki czasowy
                \item method - rodzaj metody HTTP
                \item url - adres zasobu
                \item response - kod obsługi tego żądanie
                \item byte - rozmiar odpowiedzi w bajtach
            \end{itemize}

            Poniżej znajduje się fragment opisanego pliku CSV:
            \begin{lstlisting}
,host,time,method,url,response,bytes
55,007.thegap.com,805337998,GET,/shuttle/,200,957
56,007.thegap.com,805338000,GET,/icons/menu.xbm,200,527
57,007.thegap.com,805338001,GET,/icons/blank.xbm,200,509
58,007.thegap.com,805338010,GET,/shuttle/missions/,200,12283
59,007.thegap.com,805338013,GET,/icons/text.xbm,200,527
60,007.thegap.com,805338050,GET,/icons/image.xbm,200,509
61,007.thegap.com,805338117,GET,/,200,7067
            \end{lstlisting}
        }

        \subsection{Wstępne przetworzenie pliku} {
            Do przetworzenia pliku z logami został stworzony program w języku Python z
            wykorzystaniem biblioteki \textit{Pandas}. Posiada on dwa tryby działania,
            pierwszym z nich jest tryb filtrowania, którego zadaniem jest wybór
            pierwszych 200 000 rekordów o metodzie GET, o kodzie statusu 200 oraz
            usuniecię rekordów zawierających odwołanie do plików graficznych
            (jpg, gif, bmp, xmb itp.). Samo uruchomienie dokonuję sie poprzez
            przekazanie flagi \textit{-f}, a całe polecenia wygląda następująco
            \textit{python main.py -f}.

            Drugim trybem działania programu jest tryb ekstrakcji danych, do jego
            wykorzystania jest wymaga najpierw skorzystanie z trybu filtrowania. Tworzy on
            listę najczęściej odwiedzanych stron, progiem odwiedziń była wartość 0,5\%
            oraz listę unikalnych użytkowników. Na tej podstawie została przeprowadzona
            indentyfikacja użytkowników oraz ich sesji. Sam limit zakończenia sesji
            był zdefiniowany na 600 sekund czyli 10 minut, jeśli przez ten czas
            użytkownik nie wykonał kolejnego żądania HTTP sesja była uznawana za
            zakończoną. Dla sesji zostały wyodrębnione następujące parametry:
            \begin{itemize}
                \item Czas sesji w sekundach
                \item Liczba działań w czasie sesji
                \item Przeciętny czas na stronę
                \item Zmienne flagowe dla najpopularniejszych stron
            \end{itemize}
            gdzie przeciętny czas na stronę jest to czas trwania sesji podzielony przez
            ilość odwiedzanych stron - 1.

            Samo uruchomienie dokonuję sie poprzez przekazanie flagi\textit{-g}, a całe
            polecenia wygląda następująco \textit{python main.py -g}. Efektem działania
            tego trybu jest wygenerowanie 5 plików o rozszerzeniu\textbf{.arff} o
            następujących nazwach:
            \begin{itemize}
                \item \textit{extracted\_users.arff}
                \item \textit{extracted\_sessions.arff}
                \item \textit{extracted\_user\_pages.arff}
                \item \textit{extracted\_sessions\_pages.arff}
                \item \textit{extracted\_sessions\_numeric.arff}
            \end{itemize}
        }
    }

    \section{Wiersze sterujące plików \textbf{arff}} {
        Sekcja przedstawione poniżej prezentują wiersze sterujące plików na, których
        będzie przeprowadzana analiza w programie Weka\cite{weka}.

        \subsection{Plik użytkowników} {
        % @formatter:off
            \begin{lstlisting}
@relation extracted_users
@attribute /ksc.html {True, False}
@attribute /finance/main.htm {True, False}
@attribute / {True, False}
@attribute /shuttle/countdown/ {True, False}
@attribute /shuttle/missions/missions.html {True, False}
@attribute /shuttle/missions/sts-69/mission-sts-69.html {True, False}
@attribute /procurement/procurement.html {True, False}
@attribute /htbin/cdt_main.pl {True, False}
@attribute /history/apollo/apollo.html {True, False}
@attribute /history/history.html {True, False}
@attribute /shuttle/missions/sts-70/mission-sts-70.html {True, False}
@attribute /history/apollo/apollo-13/apollo-13-info.html {True, False}
@attribute /history/apollo/apollo-13/apollo-13.html {True, False}
@attribute /shuttle/countdown/liftoff.html {True, False}
@attribute /shuttle/missions/sts-71/images/images.html {True, False}
@attribute /shuttle/technology/sts-newsref/stsref-toc.html {True, False}
@attribute /shuttle/countdown/countdown.html {True, False}
@attribute /shuttle/missions/sts-71/mission-sts-71.html {True, False}
@attribute /history/apollo/apollo-13/ {True, False}
@attribute /facilities/lc39a.html {True, False}
@attribute /shuttle/missions/sts-70/images/images.html {True, False}
@attribute /history/apollo/apollo-13/images/ {True, False}
@attribute /htbin/cdt_clock.pl {True, False}
@attribute /whats-new.html {True, False}
@attribute /shuttle/technology/sts-newsref/sts_asm.html {True, False}
@attribute /shuttle/missions/sts-73/mission-sts-73.html {True, False}
@attribute /history/apollo/apollo-13/movies/ {True, False}
@attribute /history/apollo/apollo-13/sounds/ {True, False}
@attribute /shuttle/countdown/lps/fr.html {True, False}
            \end{lstlisting}
        % @formatter:on
        }

        \subsection{Plik sesji} {
        % @formatter:off
            \begin{lstlisting}
@relation extracted_sessions
@attribute duration integer
@attribute requests_count integer
@attribute average_request_duration real
@attribute /ksc.html {True, False}
@attribute /finance/main.htm {True, False}
@attribute / {True, False}
@attribute /shuttle/countdown/ {True, False}
@attribute /shuttle/missions/missions.html {True, False}
@attribute /shuttle/missions/sts-69/mission-sts-69.html {True, False}
@attribute /procurement/procurement.html {True, False}
@attribute /htbin/cdt_main.pl {True, False}
@attribute /history/apollo/apollo.html {True, False}
@attribute /history/history.html {True, False}
@attribute /shuttle/missions/sts-70/mission-sts-70.html {True, False}
@attribute /history/apollo/apollo-13/apollo-13-info.html {True, False}
@attribute /history/apollo/apollo-13/apollo-13.html {True, False}
@attribute /shuttle/countdown/liftoff.html {True, False}
@attribute /shuttle/missions/sts-71/images/images.html {True, False}
@attribute /shuttle/technology/sts-newsref/stsref-toc.html {True, False}
@attribute /shuttle/countdown/countdown.html {True, False}
@attribute /shuttle/missions/sts-71/mission-sts-71.html {True, False}
@attribute /history/apollo/apollo-13/ {True, False}
@attribute /facilities/lc39a.html {True, False}
@attribute /shuttle/missions/sts-70/images/images.html {True, False}
@attribute /history/apollo/apollo-13/images/ {True, False}
@attribute /htbin/cdt_clock.pl {True, False}
@attribute /whats-new.html {True, False}
@attribute /shuttle/technology/sts-newsref/sts_asm.html {True, False}
@attribute /shuttle/missions/sts-73/mission-sts-73.html {True, False}
@attribute /history/apollo/apollo-13/movies/ {True, False}
@attribute /history/apollo/apollo-13/sounds/ {True, False}
@attribute /shuttle/countdown/lps/fr.html {True, False}
            \end{lstlisting}
        % @formatter:on
        }
    }

    \section{Analiza klastrowa} {
        \subfile{section/clustering.tex}
    }

    \section{Reguły asocjacyjne} {
        \subfile{section/association.tex}
    }

    \begin{thebibliography}{0}
        \bibitem{dataset}{https://ftims.edu.p.lodz.pl/mod/folder/view.php?id=97337}
        \bibitem{weka}{https://www.cs.waikato.ac.nz/ml/weka/}
    \end{thebibliography}

\end{document}
