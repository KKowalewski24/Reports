\documentclass[../main.tex]{subfiles}

\begin{document} {

    \subsection{Scenariusz bazowy}
    \label{chapter7:mozliwosci_zagrozenia:bazowy} {
        Produkt \textit{Smart Voice Access} z~kwartału na kwartał przynosi coraz
        większe zyski a~jego popularność wzrasta w sposób umiarkowany. Czas poświęcony
        na stworzenie portalu oraz stworzenie oprogramowania zaczyna się zwracać
        i~zaczyna przynosić akceptowalne zyski.
    }

    \subsection{Scenariusz optymistyczny}
    \label{chapter7:mozliwosci_zagrozenia:optymistyczny} {
        Produkt \textit{Smart Voice Access} dzięki parunastu zleceniom dla
        deweloperów bardzo szybko się spłacić i~zaczął przynosić duże zyski. Poprzez
        wiele instalacji w~nowych budynkach i~odbywające się imprezy tzn "parapetówki"
        produkt zyskał wielką popularność ponieważ znakomita część znajomych
        zaproszonych na te imprezy wykazała chęć zakupu oraz instalacji we własnym domu
        oraz oddziałach firm.
    }

    \subsection{Scenariusz pesymistyczny}
    \label{chapter7:mozliwosci_zagrozenia:pesymistyczny} {
        Produkt \textit{Smart Voice Access} nie jest chętnie wybierany przez
        developerów i~nie zyskał dużej popularności. Sami użytkownicy indywidualni też
        nie wykazują dużego zainteresowania przez co przedsięwzięcie nie jest rentowne.
        Przy utrzymywania się takie stanu rzeczy należy zaprzestanie sprzedaż i~jedynie
        dawać wsparcie dla osób, które zakupiły produkt. Same straty finansowe nie będą
        duże gdyż głównie założyciele inwestowali w~to swój prywatny czas a~nie żywą
        gotówkę.
    }
}
\end{document}
