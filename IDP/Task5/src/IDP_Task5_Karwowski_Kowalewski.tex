\documentclass{classrep}
\usepackage[utf8]{inputenc}
\frenchspacing

\usepackage{graphicx}
\usepackage[usenames,dvipsnames]{color}
\usepackage[hidelinks]{hyperref}
\usepackage{lmodern}
\usepackage{placeins}
\usepackage{url}
\usepackage{amsmath, amssymb, mathtools}
\usepackage{listings}
\usepackage{fancyhdr, lastpage}
\usepackage{subfiles}
\usepackage{ifthen}

\pagestyle{fancyplain}
\fancyhf{}
\renewcommand{\headrulewidth}{0pt}
\cfoot{\thepage\ / \pageref*{LastPage}}

% In order to change person, change value of variable
\newboolean{is_karwowski}
\setboolean{is_karwowski}{true}

%--------------------------------------------------------------------------------------%
\studycycle{Applied Information Technology, 2 cycle}
\coursesemester{II}

\coursename{Soft Computing Laboratory}
\courseyear{2021/2022}

\courseteacher{dr inż. Kamil Stokfiszewski}
\coursegroup{Wednesday, 8:30}

\author{%
    \ifthenelse{\boolean{is_karwowski}}
    {\studentinfo[239671@edu.p.lodz.pl]{Jan Karwowski}{239671}\\}
    {\studentinfo[239676@edu.p.lodz.pl]{Kamil Kowalewski}{239676}\\}
}

\title{Assignment 5.: Kohonen Network for image compression}

\begin{document}
    \maketitle
    \thispagestyle{fancyplain}

    \tableofcontents
    \newpage

    \section{Main goal} \label{main_goal} {
        The main goal of this task is to prepare implementation of Kohonen network for
        image compression. The images themselves should use 8-bit grayscale.
    }

    \section{Theoretical background} \label{theory} {
        Kohonen Network is a kind of self organizing map, where neurons and input values represent
        points (vectors) in N-dimensional space. The purpose of this model is to represent or
        reproduce shape and distribution of input values using much more lower number of points. I.e
        this algorithm groups input values into clusters (clusterize).

        Kohonen Network consists of a single layer of neurons. It is trained in unsupervised manner
        (without labels), and the training is based on simple rule \emph{winner takes all}.  This
        means, that for each input pattern only the most activated neuron's weights are modified.
        Purpose of this weights change is to make the winner neuron closer (in some sens, e.g.
        euclidean) to the input pattern.

        Getting down to the possible implementation details, neuron's activation should depends on
        \emph{distance} to input pattern. If input patterns and weights are normalized, this
        distance (or rather similarity) could be calculated as a simple dot product. If there is no
        normalization then euclidean distance could be used. After finding the nearest neuron
        (winner) for the particular pattern, neuron's weights are modified according to the
        following equation:
        \begin{equation}
            w_{i} = w_{i} + \eta (x_{i} - w_{i})
        \end{equation}
        where $w_{i}$ and $x_{i}$ are the i-th element of weight and input vectors respectively,
        $\eta$ is a learning rate.

        To compress an image using described algorithm, it should be divided to a sequence of
        random, relatively small crops (e.g. 4x4 or 16x16). Such a sequence constructs training set,
        and as a result of training Kohonen Network these crops are grouped into clusters.  To
        compress an image it should be divided to chunks and these chunks should be treated as input
        patterns and replaced by weights of the most activated neuron. When the network was trained
        without normalization this substitution is enough. When weights and input
        patterns are normalized, then pixel intensity information is lost and should be
        remembered for each chunk during compression.
    }

    \section{Implementation} \label{implementation} {
        Created program consists of two main modules. The first one is a pure Kohonen Network
        implementation, the second one is responsible for an image compression. Ours Kohonen Network
        can work in two modes - with and without normalization, which implies small differences in
        implementation. When the network is created its weights are initialized and all the input
        patterns are remembered for further training. If in \emph{normalize} mode, these two
        matrices are normalized. To train a network its \emph{train\_step} method should be called
        iteratively. Within this function single training step is proceed - for each input pattern
        winner neuron is found and its weights are updated. Winner selection depends on the running
        mode. If in \emph{normalize} mode, then dot product is calculated and the most activated
        neuron is a winner. If not in \emph{normalize} mode, then euclidean distance to each neuron
        is calculated and the nearest one is a winner. After that, dead neurons, which are not
        modified for all the input patterns in a single training step, are randomly initialized.
        Number of dead neurons and max winner weight's modification are remembered to defined stop
        constraint. Our algorithm stops if there is no dead neurons and maximum winner weight's
        modification is lower then $0.00001$ part of input patterns' min-max values range.
        Optionally, if in \emph{normalize} mode, after each training step neurons' weights are
        normalized.

        To compress an image it is read into memory in a grayscale mode. Then given number of
        random crops in given shape are extracted from the image. These random crops build dataset,
        which Kohonen Network is trained on. After training process, given number of neurons is
        available to use. Image is splitted into crops and each crop is replaced by weights of
        neuron, which is the most activated in response to the image crop. This is how decompressed
        image view is simulated. Additionally PSNR value and compression ratio are calculated.
    }

    \section{Experiments and results} \label{results} {
        \ifthenelse{\boolean{is_karwowski}}
        {\subfile{section/karwowski_results.tex}}
        {\subfile{section/kowalewski_results.tex}}
    }

    \section{Summary and conclusions} \label{summary} {
        \ifthenelse{\boolean{is_karwowski}}
        {\subfile{section/karwowski_summary.tex}}
        {\subfile{section/kowalewski_summary.tex}}
    }

    \begin{thebibliography}{0}
        % @formatter:off
        \bibitem{instruction}{Labolatory instruction, URL: https://ftims.edu.p.lodz.pl/pluginfile.php/75444/\\mod\_resource/content/1/soft\_comp\_lab\_05\_KOHONEN.pdf}
        % @formatter:on
    \end{thebibliography}

\end{document}
