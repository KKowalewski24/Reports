\documentclass{classrep}
\usepackage[utf8]{inputenc}
\frenchspacing

\usepackage{graphicx}
\usepackage[usenames,dvipsnames]{color}
\usepackage[hidelinks]{hyperref}
\usepackage{lmodern}
\usepackage{placeins}
\usepackage{url}
\usepackage{amsmath, amssymb, mathtools}
\usepackage{listings}
\usepackage{fancyhdr, lastpage}
\usepackage{subfiles}

\pagestyle{fancyplain}
\fancyhf{}
\renewcommand{\headrulewidth}{0pt}
\cfoot{\thepage\ / \pageref*{LastPage}}

\definecolor{codegreen}{rgb}{0,0.6,0}
\definecolor{codegray}{rgb}{0.5,0.5,0.5}
\definecolor{codepurple}{rgb}{0.58,0,0.82}
\definecolor{backcolour}{rgb}{0.95,0.95,0.92}

\lstdefinestyle{mystyle}{
    commentstyle=\color{codegreen},
    keywordstyle=\color{magenta},
    numberstyle=\tiny\color{codegray},
    stringstyle=\color{codepurple},
    basicstyle=\ttfamily\footnotesize,
    breakatwhitespace=false,
    breaklines=true,
    captionpos=b,
    keepspaces=true,
    numbers=left,
    numbersep=5pt,
    showspaces=false,
    showstringspaces=false,
    showtabs=false,
    tabsize=2
}

\lstset{style=mystyle}

%--------------------------------------------------------------------------------------%
\studycycle{Informatyka stosowana, studia dzienne, II st.}
\coursesemester{II}

\coursename{Eksploracja danych internetowych}
\courseyear{2021/2022}

\courseteacher{dr inż. Krzysztof Myszkorowski}
\coursegroup{poniedziałek, 12:15}

\author{%
    \studentinfo[239671@edu.p.lodz.pl]{Jan Karwowski}{239671}\\
    \studentinfo[239676@edu.p.lodz.pl]{Kamil Kowalewski}{239676}\\
}

\title{Zadanie 3.: System rekomendacji stron internetowych}

\begin{document}
    \maketitle
    \thispagestyle{fancyplain}

    \tableofcontents
    \newpage

    \section{Cel} {
        Celem zadania było przeprowadzenie rekomendacji stron internetowych, które
        powinien odwiedzic losowo wygenerowany użytkownik. Sam system działa na
        podstawie analizy aktywności innych użytkowników, dzięki której jest w stanie
        zarekomendować strony dla danego użytkownika.
    }

    \section{Zbiór danych}
    \label{dataset_description} {
        Do tego zadania został wykorzystany zbiór danych przygotowany uprzednio do
        zadania numer 1. Jest on zbudowany na podstawione cytowanego zbioru danych
        \cite{dataset}, poprzez dokonanie wyodrebniania sesji i użytkowników. W tym
        przypadku został wykorzystany plik o nazwie \textit{extracted\_user\_pages.arff}.
        Zawiera on użytkowników i flagi odwiedzonych stron, jego wiersze sterujące
        zostały przedstawione poniżej:
    % @formatter:off
        \begin{lstlisting}
@relation extracted_user_pages
@attribute /ksc.html {True, False}
@attribute / {True, False}
@attribute /shuttle/countdown/ {True, False}
@attribute /shuttle/missions/missions.html {True, False}
@attribute /shuttle/missions/sts-69/mission-sts-69.html {True, False}
@attribute /htbin/cdt_main.pl {True, False}
@attribute /finance/main.htm {True, False}
@attribute /shuttle/countdown/liftoff.html {True, False}
@attribute /history/history.html {True, False}
@attribute /history/apollo/apollo.html {True, False}
@attribute /history/apollo/apollo-13/apollo-13.html {True, False}
@attribute /shuttle/missions/sts-70/mission-sts-70.html {True, False}
@attribute /shuttle/missions/sts-71/mission-sts-71.html {True, False}
@attribute /shuttle/missions/sts-71/images/images.html {True, False}
@attribute /shuttle/technology/sts-newsref/stsref-toc.html {True, False}
@attribute /htbin/cdt_clock.pl {True, False}
@attribute /shuttle/countdown/countdown.html {True, False}
@attribute /facilities/lc39a.html {True, False}
@attribute /shuttle/missions/sts-70/images/images.html {True, False}
@attribute /history/apollo/apollo-13/apollo-13-info.html {True, False}
@attribute /procurement/procurement.html {True, False}
@attribute /shuttle/missions/sts-71/movies/movies.html {True, False}
@attribute /shuttle/technology/sts-newsref/sts_asm.html {True, False}
@attribute /history/apollo/apollo-11/apollo-11.html {True, False}
@attribute /facts/faq04.html {True, False}
@attribute /htbin/wais.pl {True, False}
@attribute /shuttle/missions/sts-70/movies/movies.html {True, False}
@attribute /images/ {True, False}
@attribute /shuttle/resources/orbiters/endeavour.html {True, False}
@attribute /shuttle/missions/sts-73/mission-sts-73.html {True, False}
@attribute /whats-new.html {True, False}
@attribute /software/winvn/winvn.html {True, False}
@attribute /facilities/tour.html {True, False}
@attribute /elv/elvpage.htm {True, False}
@attribute /shuttle/missions/sts-69/images/images.html {True, False}
@attribute /shuttle/countdown/lps/fr.html {True, False}
        \end{lstlisting}
    % @formatter:on
    }

    \section{Opis metody} {
        Na pliku \textit{extracted\_user\_pages.arff} opisanym w sekcji
        \ref{dataset_description} została wykonana klasteryzacja w programie
        Weka\cite{weka} przy pomocy algorytmu K-średnich (ang. K-Means) analogicznie
        jak w przypadku zadania pierwszego. W tym przypadku wykonaliśmy to dla trzech
        wartości parametru \emph{K} a mianowicie \emph{K=4}, \emph{K=7}, \emph{K=10}.

        W programie użytkownik jest reprezentowany jako wektor zawierające wartości
        typu Boolean czyli \emph{True} albo \emph{False}. Wartości te reprezentują czy
        poszczególne strony zostały przez niego odwiedzone.

        Do wyliczania podobieństwa użytkownika do poszczególnych grup został
        wykorzystany współczynnik podobieństwa Jaccarda, jego wzór został przedstawiony
        poniżej:

        $$
        J(A, B) = \frac{|A \cap B|}{|A \cup B|}
        $$

        gdzie w przypadku tej metody \emph{A} oznacza zbiór stron odwiedzonych przez
        użytkownika natomiast \emph{B} to zbiór zawierający strony odwiedzone przez
        większość użytkowników danego klastra. Na podstawie wyliczonej wartości należy
        wybrać ten klaster gdzie wartość jest najbliższa wartości~\emph{1} ponieważ
        wartość~\emph{1} oznacza identyność zbiorów.
    }

    \section{Wyniki}
    \label{results} {
        \subfile{section/results.tex}
    }

    \section{Dyskusja} {
        Ze względu na dosyć specyficzny rozkład danych, większość ze stron nie została
        odwiedzona przez żadnego użytkownika. Potwierdzeniem tego jest fakt, że
        najbardziej liczny klaster dla wyników w sekcji \ref{results} zawiera wszystkie
        flagi ustawione na wartość \textit{False}. W przypadku wyników dla sekcji
        \ref{results_k_4} oraz \ref{results_k_7} pozostałe klastry cechowały się
        zerowym dopasowaniem natomiast dla sekcji \ref{results_k_10} udało się uzyskać
        dopasowanie z zakresu \textit{0.06} a \textit{0.07}. Niestety są to klastry
        bardzo mało liczne i jest to poziom od \textit{1\%} do \textit{2\%}. Ze względu
        na małą różnorodność uzyskano niskie wartości współczynnika podobnieństwa
        Jaccarda.
        Można pokusić się o teorię, że zwiększenie liczby klastrów do 10 poprawiło
        wyniki natomiast nadal przez obciążenie oraz rozkład danych w tym zbiorze
        wyniki nie są znakomite. Kolejną sprawą jaka miała wpływ na wyniki jest wybrany
        użytkownik, został on przed rozpoczęciem eksperymentów wygenerowany losowo i
        użyty do wszystkich eksperymentów. Możliwe, że przy innej losowej konfiguracji
        flag stron jakie odwiedził użytkownik wyniki były lepsze. Co warto wspomnieć w
        czasie implementowania systemu rekomendacyjnego był on testowany i np dla
        użytkownika, który za każdym razem był losowo generowany dla 3 wcześniej
        wspomnianych wartości udawało się uzyskać znacznie lepsze wyniki ale tylko dla
        danej liczby klastrów i po wykorzystaniu tego użytkownika dla innej liczby
        klastrów wyniki wychodziły bardzo słabe tzn. nie odnajdywało rekomendowanych
        stron a współczynnik podobieństwa Jaccarda był bardzo niski.
    }

    \begin{thebibliography}{0}
        \bibitem{dataset}{https://ftims.edu.p.lodz.pl/mod/folder/view.php?id=97337}
        \bibitem{weka}{https://www.cs.waikato.ac.nz/ml/weka/}
    \end{thebibliography}

\end{document}
