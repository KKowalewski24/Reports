\documentclass[../main.tex]{subfiles}

\begin{document} {

    Podstawowym produktem oferowanym przez naszą firmę jest nowoczesny system dostępu
    do budynku. Może być wykorzystany przy zabezpieczaniu zarówno domków
    jednorodzinnych, bloków, kamienic jak i~całych kompleksów i~osiedli mieszkalnych.
    Produkt przeznaczony jest zatem dla bardzo dużego grona odbiorców, na których
    składają się tak samo klienci indywidualni jak i~przedsiębiorcy (np. deweloperzy).
    Wszyscy, którym zależy na wygodnym dostępie do własnego mieszkania, biura, garażu
    czy jakiegokolwiek innego miejsca, do którego dostęp jest ograniczony, może być
    zainteresowany naszą ofertą.

    \subsection{Idea produktu}
    \label{chapter1:opis_produktu:idea_produktu} {
        Podstawową ideę produktu można przedstawić na przykładzie najprostszego
        użytkownika, jakim jest pojedyncza osoba fizyczna. Użytkownik taki, ma potrzebę
        kontrolować dostęp do swojego mieszkania. W~jego bloku (jak w każdym w ogóle
        miejscu) jest możliwość zamontowania proponowanego przez nas rozwiązania. W
        sytuacji, gdy człowiek taki chce dostać się na klatkę schodową, musi jedynie
        powiedzieć cokolwiek do znajdującego się przy drzwiach wejściowych mikrofonu.
        Jeśli jego głos, wcześniej zarejestrowany w~systemie, zostanie rozpoznany,
        drzwi się odblokują i~będzie można je otworzyć. W przeciwnym wypadku nic się
        nie stanie. Jeżeli na klatkę chce się dostać człowiek obcy, mówi do mikrofonu
        imię i nazwisko, lub numer mieszkania osoby, którą chce odwiedzić. Głos zostanie 
        zinterpretowany i nawiązane zostanie połączenie telefoniczne z numerem telefonu
        zarejestrewanym dla danego lokalu. Taka forma kontroli dostępu niesie ze sobą 
        bardzo wiele korzyści, spośród najważniejszych należy wymienić:
        \begin{itemize}
            \item nie trzeba posiadać żadnego rodzaju fizycznego klucza
            \item nie trzeba pamiętać hasła
            \item aby uzyskać dostęp nie trzeba w ogóle używać rąk (np. kiedy ktoś
            niesie ciężkie torby z zakupami nie musi wpisywać kodu ani szukać kluczy),
            wystarczy powiedzieć cokolwiek do mikrofonu
            \item nie trzeba instalować żadnych dodatkowych urządzeń w mieszkaniach
            \item mieszkaniec jest powiadamiany o obecności gości, nawet kiedy
                nie ma go w domu
        \end{itemize}
    }

    \subsection{Budowa produktu i jego elementy}
    \label{chapter1:opis_produktu:budowa_produktu} {
        Oferowany system co do zewnętrznego wyglądu i~struktury fizycznej instalacji
        nie różni się zbytnio od najprostszego domofonu analogowego, który można
        znaleźć zarówno w nowoczesnych jak i~starszych budynkach. Różnica polega na
        tym, że w~przypadku naszego produktu nie ma potrzeby montowania tzw. unifonów.
        Decyzja o~otworzeniu drzwi zawsze jest podejmowana przez system. Nie ma również
        potrzeby umieszczania żadnych przycisków przy drzwiach wejściowych. Na kompletny 
        produkt składają się następujące elementy.
        \begin{itemize}
            \item mikrofony umieszczone w~bramach i~drzwiach wejściowych (małe
            urządzenia średnicy ok. 3 cm)
            \item centrala, czyli zasilacz wraz z~głównym komputerem, znajdujące się
            w~jednym miejscu na zabezpieczonym terenie (niewielka skrzynka z~anteną,
            rozmiarem i~wagą zbliżona do komputera stacjonarnego)
            \item system połączeń między mikrofonami a~centralą w postaci prostych
            wielożyłowych kabli
            \item centralny serwer obliczeniowy, z~którym łączą się centrale,
            znajdujący się pod kontrolą dostawcy produktu (może znajdować się w chmurze
            i~nie mieć postaci fizycznej)
        \end{itemize}
    }

    \subsection{Rozwiązania konkurencyjne}
    \label{chapter1:opis_produktu:rozw_konkurencyjne} {
        Podstawowym, znanym wszystkim konkurencyjnym rozwiązaniem jest klasyczny
        domofon lub trochę nowocześniejszy wideofon. Pierwsze rozwiązanie zapewnia
        połączenie głosowe z~osobą znajdującą się na zabezpieczonym terenie, drugie
        natomiast dodaje do tego połączenie wizualne (wzrokowe). Połączenia te mają na
        celu możliwość identyfikacji osoby przychodzącej przez osobę będącą już na
        zabezpieczonym terenie i~udzielenie przez tą drugą zdalnie dostępu. Dodatkowo
        zazwyczaj istnieje możliwość wpisania czterocyfrowego kodu, który pozwala
        uzyskać dostęp bez czynnego udziału osoby “z~wewnątrz”. Ponadto dla
        bezpieczeństwa ostateczną formą dostania się na zabezpieczony teren jest użycie
        zwykłego, fizycznego klucza w połączeniu ze zwykłym, fizycznym zamkiem. To
        standardowe rozwiązanie jest przystosowane przede wszystkim aby ułatwić szybkie
        i~wygodne podjęcie decyzji o udzieleniu dostępu przez osobę znajdującą się już
        na terenie z dostępem ograniczonym. Jeżeli chodzi o przypadek, kiedy ktoś chce
        uzyskać dostęp do miejsca, w którym nikogo nie ma, system ten okazuje się
        znacznie mniej pomysłowy i~sprowadza się do użycia klucza fizycznego, bądź też
        cyfrowego.

        Rozwiązanie proponowane przez naszą firmę skupia się na tym drugim przypadku,
        kiedy to ktoś próbuje uzyskać dostęp do miejsca, w~którym nikt się w danej
        chwili nie znajduje. Zaprojektowany przez nas system wielokrotnie przyspiesza
        i~ułatwia ten proces, w~stosunku do rozwiązań obecnych na rynku.
    }

    \subsection{Proces produkcji i instalacji}
    \label{chapter1:opis_produktu:produkcja} {
        Z~punktu widzenia biznesowego na nasz produkt składają się dwa główne podprodukty:
        \begin{enumerate}
            \item produkt fizyczny - wszystkie urządzenia: centrala i~mikrofony oraz
            okablowanie
            \item oprogramowanie - system działający na centralnym serwerze,
            odpowiedzialny za rozpoznawanie głosu i wykonywanie połączeń telefonicznych
        \end{enumerate}

        Aby wyprodukować gotowy system, możliwy do zainstalowania na danym terenie
        będzie potrzebne skorzystanie z~usług innych firm. Po pierwsze układy
        elektryczne i~ich obudowy (mikrofony i centrale), zaprojektowane przez naszych
        specjalistów, muszą być wytworzone przez zewnętrzną firmę a~następnie dostarczone
        do naszej siedziby. Po drugie centralny serwer obliczeniowy będzie miał postać
        wirtualną i~będzie zarządzany przez firmę hostingową (np. Amazon). Na końcu
        sama instalacja produktu na danym terenie wymaga obecności pracowników
        fizycznych, którzy również są zatrudnieni przez pewną zewnętrzną firmę, która
        może zmieniać się w czasie i~wobec której nie stawia się żadnych
        specjalistycznych wymagań.
    }
}
\end{document}
