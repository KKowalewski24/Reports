\documentclass[../main.tex]{subfiles}

\begin{document} {

    \subsection{Pozycjonowanie firmy w internecie}
    \label{chapter4:plan_marketingowy:pozycjonowanie} {
        Ze względu na kompetencje zespołu zarządzającego, które zostały przedstawione
        w~sekcji \ref{chapter2:zespol_zarzadzajacy} zostanie utworzony portal
        internetowy, którego celem zarówno będzie reklama jak i~możliwość zamówienia
        produktu.

        Zadaniem części reklamowej będzie przedstawienie produktu od jak najlepszej
        strony poprzez pokazanie jego zalet w~interaktywnej formie. Będą tam również
        przedstawiona firmy oraz osoby, które zdecydowały się na zakup i~chciały
        podzielić się opinia na temat tego produktu. Będzie również przedstawiona
        historia firmy, jej założyciele oraz formularz kontaktowych w~przypadku pytań
        ze strony klientów.

        Druga część portalu internetowego będzie zawierać wirtualny sklep w, którym
        będzie istniała możliwość złożenia zamówienia, podania wybranej specyfikacji
        i~możliwość wykonania zapłaty lub przelania zadatku, gdzie reszta kwoty zostanie
        przekazana po wykonaniu usługi montażu urządzenia.
    }

    \subsection{Pozyskiwanie klientów}
    \label{chapter4:plan_marketingowy:pozyskiwanie_klientow} {
        Ze względu na szerokie znajomości w~branży budowlano-deweloperskiej Kamila
        Kowalewskiego, jednej z~osób z~zespołu zarządzającego, znalezienie szerokiego
        grona odbiorców nie będzie stanowić problemu gdyż podczas budowy montowane są
        takie urządzenia w pokaźnej liczbie, dzięki czemu można się spodziewać stałej
        i~wysokiej liczby klientów. Co więcej dzięki portalowi internetowemu ze
        skonfigurowanym wysokim pozycjonowaniem, który został opisany w~sekcji
        \ref{chapter4:plan_marketingowy:pozycjonowanie} zdobywanie jednostkowych
        klientów nie będzie stanowiło problemu. Planujemy również w~niedalekiej
        przyszłości skorzystać z~reklam na portalach społecznościowych takich jak
        Facebook\cite{facebook} czy Instagram\cite{instagram} aby jeszcze bardziej
        zwiększyć rozpoznawalność naszej marki.
    }

    \subsection{Utrzymywanie klientów}
    \label{chapter4:plan_marketingowy:utrzymywanie_klientow} {
        W~celu utrzymania klientów z~branży budowlano-deweloperskiej będziemy stosować
        rabaty na kolejne zakupy na, których wysokość będzie wpływała liczba
        zakupywanych urządzeń. Dla klientów jednostkowych planujemy kampanie
        z~wykorzystaniem newslettera w, którym znajdowałyby się okazyjne oferty oraz
        możliwość dokupywania nowo powstałych funkcjonalności. Sama struktura cen dla
        obecnych klientów byłaby niższa.
    }

    \subsection{Struktura przychodów}
    \label{chapter4:plan_marketingowy:strukutura_przychodow} {
        Ze względu na dosyć rozbudowana architekturę rozwiązania koszty zostały
        przedstawione poniżej:\\
        \noindent\textbf{Koszty urządzeń fizycznych} - w ich skład wchodzą mikrofony
        i~okablowanie oraz koszty montażu - \textit{50 złotych}\\

        \noindent\textbf{Koszt centrali} - jest on pokrywany przez pierwszą osobę
        montującą system w~danym budynku, w~przypadku gdy kolejne osoby się zdecydują
        pierwszej osobie są zwracane proporcjonalnie koszty zależnie od liczby osób
        korzystających z~technologii, w cenie tej jest wliczony również oraz koszt
        montażu - \textit{350 złotych}.\\

        \noindent\textbf{Comiesięczny abonament} - ze względu na wykorzystanie usług
        chmurowych musi być pobierana opłata za ich użycie - koszt miesięczny dla
        jednego urządzenia to \textit{10 złotych}.\\

        W~przypadku cen dla odbiorców z~branży budowlano-deweloperskiej przy koszty
        kalkulacji dla jednego budynku składają się na 1x centrala, mikrofony
        i~okablowanie zależnie od liczby klatek schodowych w~danym budynku oraz
        doliczane do czynszu kwota za comiesięczny abonament.
    }

    \subsection{Opis strategii sprzedaży}
    \label{chapter4:plan_marketingowy:strategia_sprzedazy} {
        Zgodnie z~założeniami opisanymi w~sekcjach
        \ref{chapter4:plan_marketingowy:pozycjonowanie} oraz
        \ref{chapter4:plan_marketingowy:pozyskiwanie_klientow} sprzedaż produktu będzie
        realizowano poprzez portal sprzedaży z~modułem płatności. Co tego celu
        w~pierwszych miesiącach działalności nie będziemy wykorzystywać dodatkowych osób.
        Istnieje możliwość, że po pewnym czasie zostanie zatrudniony przedstawiciel
        handlowy, którego zadaniem będzie zdobywanie jeszcze większej liczby klientów
        i~wizytacja na różnego rodzaju targach.
    }
}
\end{document}
