\documentclass[../main.tex]{subfiles}

\begin{document} {

    Według danych GUS w ostatnim roku powstało prawie 200 tysięcy nowych mieszkań. Każde musi zostać
    zabezpieczone przed niepowołanym dostępem i każde musi być gotowe przyjąć swojego lokatora
    obładowanego zakupami, dobijającego się na klatkę schodową w deszczu. Najlepiej by było, gdyby
    drzwi otwierały się same, kiedy rozpoznają, że zbliża się domownik. Zwierzęta rozpoznają siebie
    nawzajem wykorzystując wszystkie pięć zmysłów. Maszyna również może to zrobić, ale najtańszy ze
    wszystkich jest słuch. Prosty mikrofon kosztuje dziesiątki razy mniej, niż kamera czy czujnik
    linii papilarnych, nie wspominając o technologiach rozpoznających zapach. Zapewnijmy zatem
    ludziom dostęp do mieszkań na podstawie ich słów i barwy głosu, które w pełni identyfikują.

    Naprzeciw tym śmiałym oczekiwaniom wychodzi nasze przedsiębiorstwo oferujące nowatorski produkt
    w postaci inteligentnego systemu dostępu o nazwie \emph{Smart Voice Access}. Pod tą nazwą kryje
    się dwóch ambitnych ludzi, którzy po kilkunastu latach doświadczenia komercyjnego w
    szeroko rozumianej branży informatycznej chcą pójść dalej. Jan Karwowski i Kamil Kowalewski,
    absolwenci Politechniki Łódzkiej współpracują już od czasów studiów. Ich doświadczenie jest
    różne, ale w połączeniu wszechstronne i dające bardzo dobrą bazę dla tego nowego,
    technologicznego przedsięwzięcia. Jan jest specjalistą w tworzeniu inteligentnych
    rozwiązań i systemów wbudowanych, Kamil specjalizuje się w biznesie i potrafi efektywnie
    zorganizować pracę w najtrudniejszych warunkach.

    Proponowane przedsięwzięcie posiada dwie cechy, które zapewniają o jego powodzeniu. Po pierwsze
    oferowany produkt jest nieprawdopodobnie wręcz tani, w stosunku do rozwiązania obecnego na
    rynku. Po okresie projektowania i tworzenia produktu, wymagającego dużo pracy od jego autorów,
    która przeliczy się zapewne na koszty rzędu kilkudziesięciu tysięcy złotych nastąpi okres
    wdrożenia i utrzymania. Zadziwiające jest, że wyprodukowanie i instalacja proponowanego systemu
    w średniej wielkości bloku, mieści się w granicach tysiąca złotych. Zamontowanie pełnego systemu
    domofonowego uwzględniając każde mieszkanie to kwota kilka jeśli nie kilkanaście razy większa.

    Drugi powód to funkcjonalność. Proponowane rozwiązanie wymaga od użytkownika używania głosu aby
    dostać się do swojego domu. Nie trzeba nic pamiętać ani posiadać żadnych fizycznych obiektów
    (kluczy), które zapewniają dostęp. Każdy kto choć raz stał na deszczu lub na mrozie i szukał w
    torbie kluczy do klatki, albo usiłował wpisać kod do drzwi będąc obładowanym zakupami, bardzo
    dobrze zrozumie, że możliwość odblokowywania drzwi dowolnym okrzykiem będzie bardziej
    niż wygodna.

    Możliwości naszego produktu są większe, niż ludzie potrafią sobie wyobrazić, dlatego też
    istnieje ryzyko uznania naszego pomysłu za kolejną nowinkę technologiczną. Kluczowe zatem przy
    zyskiwaniu popularności będzie dostrzeganie naszego rozwiązania wdrożonego "u sąsiada". Wyda się
    to zapewne najpierw fascynujące i ciekawe, następnie ludzie będą dostrzegać praktyczną stronę
    rozwiązania. Oczekujemy, że pierwsze miesiące będą przynosiły małe zyski, produkt jednak z
    powodu niepodważalnej swojej przewagi nad konkurencyjnym, klasycznym domofonem, w końcu zyska
    popularność. Będzie ona rosnąć tym szybciej, im więcej ludzi zainstaluje u siebie system
    \emph{Smart Voice Access}.

    W pierwszych latach wielokrotnie zwróci się koszt pracy, włożony w przygotowanie produktu. Kiedy
    przychody ustabilizują się nastanie potrzeba rozwoju firmy. Ze względu na posiadane
    doświadczenie docelowym rynkiem pozostaną systemy dostępu oparte o ludzki głos. Być może w
    przyszłości również zajdzie potrzeba wykorzystania systemów wizualnych. Kierunek ten zależy
    jednak w większości od zastanej sytuacji rynku i rozmiaru odniesionego sukcesu, który w tym
    konkretnym przypadku jest niewątpliwy.

}
\end{document}
