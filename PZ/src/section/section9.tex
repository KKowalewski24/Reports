\documentclass[../main.tex]{subfiles}

\begin{document} {
    Plan finansowy w~tym przypadku można rozpatrywać dwojako, może zawierać same koszty
    poniesione przez nas lub może zawierać czas przeliczony na pieniądze poświęcony na
    stworzenie oprogramowania do produktu jak i~portalu internetowego.

    W~pierwszym przypadku bazowymi kosztami to zakup centrali, mikrofonu oraz
    okablowania w~celu opracowania oprogramowania i~technologii, według aktualnych cen
    rynkowych wyniesie to \textit{180 złotych}. Opracowanie technologii pochłonie czas
    jednego z~założycieli Jana Karwowskiego i około \textit{10 złotych} za wykorzystany
    prąd. Opracowanie portalu internetowego pochłonie czas drugiego z~założycieli
    Kamila Kowalewskiego i~około \textit{10 złotych} za wykorzystany prąd.

    Koszty wzrosną wraz z~pierwszymi zamówieniami gdyż trzeba będzie wynająć
    pracowników fizycznych do montażu produktu. Koszt samych części w~przypadku
    zamówień hurtowych spadnie o~\textit{30 złotych} czyli wyniesie \textit{150 złotych}.
    Biorąc pod uwagę koszty dla nabywcy przedstawione w~sekcji
    \ref{chapter4:plan_marketingowy:strukutura_przychodow} czyli łącznie
    \textit{400 złotych} i~koszty podzespołów zostanie \textit{250 złotych}. Z~każdego
    montażu osoba montująca dostanie \textit{100 złotych} a~pozostałe
    \textit{150 złotych} zostaje podzielone między dwóch założycieli. W~ciągu dnia
    czyli 8 godzin roboczych możliwa jest całkowita instalacja 4 takich urządzeń.
    Całkowita instalacja jest definiowana przez instalacje również centrali.
    W~przypadku gdy jest już zainstalowana centrala jest możliwość 20 instalacji jednego
    dnia czyli zysk wyniesie \textit{1000 złotych} natomiast koszt pozostałych
    części wynosi \textit{10 złotych} za sztukę więc dziennie mamy \textit{800 złotych}
    zysku. Takich kombinacji może być praktycznie nieskończenie, powyżej zostały tylko
    przedstawione symulacje.

    Z~samych obonamentów również będziemy czerpać korzyści. Z~naszych symulacji wynika,
    że przeciętny użytkownik wykorzysta zasoby usług chmurowych za \textit{3 złote}
    miesięcznie więc mamy na każdym użytkowniku \textit{7 złotych} miesięcznie.

    W~przypadku gdy przeliczymy swój czas poświęcony na realizację technologii nie
    wygląda to już tak kolorowo gdyż stworzenie technologii zajmie około 100 godzin co
    według stawek rynkowych będzie kosztować \textit{10000 złotych} oraz stworzenie
    portalu zajmie jakieś 80 godzin i~będzie kosztować \textit{8000 złotych}. Warto
    dodać, że sam portal internetowy w łatwy sposób można przerobić pod dane wymagania
    i~sprzedawać klientom z~innych branż w~cenie \textit{5000 złotych} za portal oraz
    wdrożenie na serwer produkcyjny natomiast technologie rozpoznawania można
    opatentować i~również czerpać z~niego korzyści finansowego.
}
\end{document}
