\documentclass[../IDP_Task5_Karwowski_Kowalewski.tex]{subfiles}

\begin{document} {

    \subsection{4x4 crop size} \label{results_4x4} {
        First of all taking look at table \ref{crop_size_4.txt} and charts
        \ref{fig:compression_4}, \ref{fig:neurons_4}, it is easy to see that PSNR value
        increases when there is more neurons and compression ratio is lower. What is
        very interesting dependency between PSNR and neurons number is rather
        exponential, this state can be confirmed by the images shown in section
        \ref{kowalewski_results_4x4}. There is a huge difference between quality of
        images with small number of neurons - 3, 6, 9 but when we take a look at
        images with big number of neurons - 60, 70, 80, 90, 100 the different is really
        small, it can be said that there is not difference for human eyes.
    }

    \subsection{8x8 crop size} \label{results_8x8} {
        The overall results are quite the same as described in section
        \ref{results_4x4} but in this case this requires more neurons to achieve the
        same results. What is more clustering image pixel space using 8x8 crop size
        takes much longer than in 4x4 case. Comparing results achieved for 100 neurons
        PSNR is higher for 4x4 that 8x8 which implies better results for 4x4 using the
        same number of neurons.
    }
}
\end{document}
