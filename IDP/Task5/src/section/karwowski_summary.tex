\documentclass[../IDP_Task5_Karwowski_Kowalewski.tex]{subfiles}

\begin{document} {
    Figures \ref{fig:4x4_images}, \ref{fig:4x4_number_of_neurons} and
    \ref{fig:4x4_compression_ratio} present results of the first series of experiments - related to
    4x4 crop size. First of all, we can observe trivial dependency: when there is more neurons and
    compression ratio is lower, then PSNR value increases. I.e. the more neurons we use the better
    is quality of decompressed image - the information lost is lower. This dependency is visible in
    all the three figures. More interesting observations include how exactly PSNR depends on number
    of neurons. In the fig. \ref{fig:4x4_number_of_neurons} we can see that this is probably
    exponentiall dependency - in higher neurons quantities there is no such a big impact on decoded
    image quality. It is also visible in fig. \ref{fig:4x4_images}, only a few beginnig thumbnails
    have very noticable quality lost in compare to the rest of them. On the other hand in fig.
    \ref{fig:4x4_images} we can see that all these images have visible quality lost in compare to
    the original image. Another observation is related to the shape of blue curve in fig.
    \ref{fig:4x4_compression_ratio}. We can notice some kind of steps when PSNR value going down.
    This behaviour is related to compression ratio calculation method - compression ratio depends on
    number of bits required to save all necessary information. Using an extra bit when storing
    compressed image results in possiblity of using many new ,,number of neurons'' values.

    The second series of experiments is related to greater crop size - 8x8. Change of this paricular
    parameter has a greate impact on image quality lost for the same number of neurons. In fig.
    \ref{fig:8x8_number_of_neurons} and \ref{fig:8x8_compression_ratio} we can observe lower values
    of PSNR in compare to the previous results. Also fig. \ref{fig:8x8_images} presents images,
    which have noticable lower quality. On the other hand, using such a big crop size results in
    much more compression ratio, as we can see in the plots. Another difference is that in fig.
    \ref{fig:8x8_compression_ratio} we can not see such a big ,,steps'' as in fig.
    \ref{fig:4x4_compression_ratio}. It is caused by the fact, that when new neuron is added then
    memory, which is required to store compressed image, increases much more then in case of 4x4
    crops, i.e. $8 \cdot 8 = 64$ new values have to be stored, instead of $4 \cdot 4 = 16$ new
    values. It worth mentioning, that clusterizing image pixel space using 8x8 crop size is much
    more time consuming. Except of these observations the second series of experiments is a source
    of very similar conlusions to the first series.
}
\end{document}
